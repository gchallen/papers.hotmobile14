\section{Sustainability Today}
\label{sec-sustainability}

We spoke to Sprint about their efforts in the area of smartphone
sustainability. Sprint has been recognized as one of the greenest companies
in the US by Newsweek's annual Green Rankings~\cite{sprintgreen-url}, and has
ambitious goals for greening mobile devices. Sprint offers users credit for
their old phones during the purchase of a new device, with the amount
depending on the phone model and condition. They aim to recover 90\% of their
users phones at end-of-life by 2017, an aggressive target given today's
industry average of recovering only 10\% and Sprint's current rate of 44\%.
Sprint's goal may also be difficult to achieve because many users choose to
keep their old phone around as a spare device, despite the pile of old spare
devices gradually filling their closet. And the user demand for new
smartphones shows no sign of abating, with Verizon, AT\&T and T-Mobile
offering new plans tailored at users that want to replace their devices even
more frequently.

When we asked Sprint specifically about what they do with devices that the
repurchase, their answer focused on enabling reuse of smartphones as
smartphones. After paying to test and, if necessary, refurbish the returned
phone, they resell it as used to a second user, either in the US or overseas.
Of the phones they have recovered from their buyback program, 80\% can be
rebranded and reused, 10\% are desirable phones but not compatible with
Sprint's network, and the final 10\% are so broken that their only value is
the \$1 or \$2 or precious metals they contain~\cite{FIXME-goldinphone}.

While creating a market for used smartphones is an appropriate first step, it
ignores what happens when the second user returns the doubly-used but still
functional device. As with other electronics, the value of smartphones drops
extremely quickly. Sprint informed us that they were offering users only \$22
for a Samsung Nexus~S~4G in good condition, three years after it sold for
\$529 unlocked. After several iterations either one of two things will
happpen: the buyback price will be too low to incentivize the user to return
the phone, or the phone will be old enough to not be attractive to users in
the market for a smartphone. At this point in the smartphone's life it
options are limited, but we believe that a significant opportunity for reuse
exists.
