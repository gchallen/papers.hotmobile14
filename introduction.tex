\section{Introduction}

Smartphone technologies are advancing rapidly, bringing new power into users
pockets and changing the way that we live and work. The rapid rate at which
consumers purchase new smartphones can be seen as primarily a response to the
rate at which this technology is improving. Short device lifetimes, while
unfortunate from a sustainability perspective, help support companies that
build and sell smartphone hardware and software. Unfortunately smartphones,
like most other electronics, are difficult to dispose of properly. Many end
up unused in desk drawers, discarded in landfills, or shipped to poor
countries where they are dangerously dismantled in an effort to extract
precious metals.

Given the potential of the smartphone to bring about transformative
technological change, it becomes difficult to reduce \textit{demand} by
arguing that consumers should hang on to outdated devices in the name of
sustainability. Instead, we propose to focus on the \textit{supply} of fully-
or partially-functional outdated devices that society currently struggles to
put to use, and explore how this growing volume of techno-trash can be
efficiently reused.

There are three reasons why the time is right for this effort. First, unlike
previous generations of ``feature phones'', the current smartphone market is
coalescing around a small set of common platforms such as Android. This
platform homogeneity reduces the burden of supporting large numbers of
discarded devices. Second, current smartphones have an attractive feature set
for many non-phone applications: size and power requirements facilitating
easy deployment, microphones and cameras allowing them to double as sensors,
touch screens for interfacing with users.

Finally, smartphones are well-integrated into the existing communication
infrastructure. They can transmit data via text messages, over WiFi networks,
and via high-speed mobile communication technologies like 3G. If WiFi is
available, no service plans are required to allow recycled smartphones to
become part of the ``Internet of Things''. And with carriers increasingly
interested in ``machine-to-machine'' applications, we expect to see
increasing service flexibility allowing discarded devices to be cheaply
connected to pervasive mobile cellular and data networks.

To provide an idea of the potential of discarded devices, the U.S.
Environmental Protection Agency (EPA) estimated that 141~million mobile
devices became ready for end-of-life management in 2009, of which only 11.7
million (8\%) were collected for recycling~\cite{epa-ewasteweb}. The
129~million phones discarded in 2009 would be enough to place an average of
\textit{200 phones} on all 600,000~bridges in the United States, or every
\textit{2 feet} on every stretch of highway in the 46,876~mile interstate
highway system.
