\section{Determining Application Utility}
\label{sec-utility}

Smartphone energy modeling provides today's users with easy access to
information about the energy used by the apps running on their device, a
feature incorporated into all major smartphone platforms. However,
prioritizing energy usage between applications requires more information than
just usage statistics. How efficient is the application? How important is it
to the user? What is the relationship between the amount of time it is
actively used and its energy consumption? How sensitive is it to
energy-driven performance degradation? None of these questions can be
answered by only measuring the amount of energy consumed. Usage must be put
into context.

Doing so requires collecting information at the platform level and so will
necessitate changes to the Android sources themselves. Fortunately, our
\PhoneLab{} testbed provides the ability to experiment with platform changes
at scale, which is not currently possible in any other way. (Application
marketplaces do not distribute platform modifications for obvious security
reasons.) We propose to investigate what kinds of information help put energy
usage in context. At a high level, our initial approach is to measure the
amount of information delivered by each application and look at how much
energy is required to do so. Examples of information delivery include
fetching a new email and presenting it to the user, as Gmail would do;
rendering the environment for a game and redrawing the screen, as Angry Birds
would do; or downloading, decoding and rendering a video, as the YouTube
application would do.

Measuring information delivery requires monitoring the interfaces that
smartphones use to deliver content, primarily the display and audio output.
We will then attempt to determine the information content of the stream,
potentially by looking at how compressible it is or using a standard
information theory metric. We may also measure the amount of interaction the
user has with the application, but it is not clear at present if this is
necessary. Simultaneously, we will be measuring the amount of energy the
application is using and provide both as inputs to the Jouler policy
framework. 

\textbf{Our overall goal is to provide a single ratio of energy to
information delivery that measures the efficiency of a broad class of
applications.} Ideally we could use this as direct input into the energy
prioritization process, rewarding applications that deliver a large amount of
content-per-byte and deprioritizing inefficient applications that consume a
great deal of energy but deliver little information.
