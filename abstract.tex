\begin{abstract}

As the rapid pace of smartphone improvements drives consumer appetites for
the latest and greatest devices, the hidden cost is millions of tons of
e-waste containing hazardous chemicals and difficult to dispose of safely.
Studies show that smartphone users are replacing their devices every
18~months, almost \textit{three times} faster than desktop
computers~\cite{computerreplacement-url,smartphonereplacement-url}, producing
millions of discarded smartphones each year that end up lying in desk
drawers, buried in landfills, or shipped to third-world countries where they
are burned to extract precious metals, a process that damages both the health
of those involved and the environment.

\sloppypar{Fortunately, the capabilities of discarded smartphones make them
ideal for reuse. Instead of ending up in a landfill, a discarded smartphone
could be integrated into a home security system or transformed into a health
care device for the elderly. In this paper, we evaluate using discarded
smartphones to replace traditional sensor network ``motes''. Compared with
motes, discarded devices have many advantages: price, performance,
connectivity, interfaces, and ease of programming. While the main question is
whether their energy consumption is low enough to enable harvesting solutions
to allow continuous operation, we present preliminary results indicating that
this may be possible.}

\end{abstract}
