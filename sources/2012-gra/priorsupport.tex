\thispagestyle{numberonly}
\chapter{Results from Prior Google Support}

Geoffrey Challen was a co-investigator on a project entitled
``\textsc{PhoneLab}: A Participatory Smartphone Cloud Testbed'' which
received \$60,994 from the Google University Award program in 2011.

We used this funding to purchase 40 Nexus S 4G smartphone from Sprint and
equip them with voice and data service. In addition, two students were
supported during the 2011--2012 academic year. Equipment and student support
were used to design a small prototype of the large open-access smartphone
testbed called \textsc{PhoneLab} which SUNY Buffalo will open in the fall of
2012. During the last nine months, our team has focused on building the
software necessary to control testbed-attached devices, log data, reliably
collect data at central servers, and manage the testbed. At this point we
have a working set of tools in continuous use managing our prototype testbed.

In addition to the core infrastructure work, a series of research projects
were initiated last fall as part of a graduate-level systems research class
that focused on smartphone research and development, and these early projects
provided valuable insights into the features that the testbed would need to
support to accelerate smartphone research. While they have yet to produce
results, several projects are ongoing.

With support from Google, Sprint, and many of our research colleagues at
other institutions, the \textsc{PhoneLab} project recently received a 3 year
1.3~million dollar Computing Research Infrastructure (CRI) grant from the
National Science Foundation. This funding will allow us to open
\textsc{PhoneLab} this fall with 250 phones and eventually scale the testbed
to 750 connected devices by 2015. A stated goal when we applied and received
money previously from Google to support \textsc{PhoneLab} was to eventually
scale it to a size that the University Awards program would have been unable
to support, and we are pleased that we have been able to use the initial seed
funding from Google to do so.
