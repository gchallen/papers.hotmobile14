%          3. Description of the work you'd like to do, and the expected
%             outcomes and results. How this relates to prior work in the area
%             (including your own, if relevant).
%              1. Differentiate from prior work
%              2. Explain how research challenges will be addressed
%              “It's hard to have a big impact without taking risks, but
%              please identify what the difficulties are likely to be and how you plan to
%              mitigate them. It may help to explain how you succeeded in addressing such
%              problems in other projects.” 

\section{Challenges}
\label{sec-technical}

There are several challenges that must be overcome in order to enable
efficient reuse of discarded smartphones. One is \textbf{power efficiency}.
Many of our imagined reuses of discarded phones place them far from the
existing power infrastructure, meaning that they will have to continuously
harvest power and conserve energy during use. The mobile systems and sensor
networking communities have been studying power conservation in many similar
settings, and we expect to be able to reuse many existing ideas while
inventing new solutions when necessary.

The second is \textbf{device capability determination and characterization.}
To efficiently reuse large numbers of discarded phones, the cost of the
transformation must be as low as possible. This cost includes the process of
identifying the phone and determining its capabilities. We expect the pool of
discarded devices to eventually include many different models of smartphone
with different features. In addition, there will be process variation between
identical phones as well as additional variation caused by phone usage.
Certain parts of the phone may not be working well; others may be entirely
broken. Matching phones with any number of intended second uses will require
new testing procedures that can accurately identify what phone features are
available and how well they are functioning.

A third challenge is \textbf{how to create a single-purpose device from a
multi-purpose phone.} Imagining the phone as a powerful building block for
new devices, we must determine what the interfaces will be that will allow
the phone core to be attached to whatever special-purpose peripherals the
device needs to function. In cases where the phone is replacing
special-purpose hardware we must determine how to transform the existing
software to run on the recycled phone. And if new software is being
developed, the problem includes the continuation of work on making smartphone
programming easier.

In addition, it may be necessary to research the structure of an appropriate
computing platform for smartphone recycling. Smartphones are considered
general-purpose computing platforms, but many of our recycling ideas put them
to more specific uses. It is possible that the existing smartphone software
platform is too general purpose to efficiently support these new classes of
recycled uses, and must be refactored to be more efficient at a small set of
core tasks.

Finally, recycled devices must be integrated into the existing communications
infrastructure. While the large number of different communication protocols
integrated into typical smartphones makes this possible, the power
requirements and deployment locations may make efficient integration
challenging. Carrier flexibility will also be necessary, since keeping the
cost of applications using recycled devices down will require new service
plans tailored to their communication patterns.
