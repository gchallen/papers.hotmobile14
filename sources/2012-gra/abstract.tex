\begin{quote}
The consumer economy loves a product that sells at a premium, wears
out quickly or is susceptible to regular improvement, and offers with each
improvement some marginal gain in usefulness.
\hfill---Jonathan Franzen
\end{quote}

\section*{Abstract}

Electronic waste is a growing problem as the rapid pace of technological
improvements drives consumer appetites for the latest and greatest devices.
The result is millions of tons of e-waste, much of it containing hazardous
chemicals and difficult to dispose of safely. Smartphones are already part of
this problem, and given the rate of progress in the smartphone technologies,
it seems reasonable to expect that consumers will dispose of these devices at
rapid rates, spurred on by new features and discounts offered by carriers.

Fortunately, the capabilities, connectedness, and platform homogeneity of
smartphones make discarded devices ideal \textit{building blocks} for many
second uses. Instead of ending up in a landfill, a discarded smartphone could
be transformed into a sensing device monitoring infrastructure or a health
care device for the elderly or disabled. \textbf{We propose to improve
smartphone sustainability by addressing the challenges inherent in turning
techno-trash into treasure.}
