\section{Idealized Components as Narrow Waste for Cross-Device Policies}

Controlling energy usage requires controlling the hardware components that
consume energy, but this is difficult to impossible when each provides a
different interface and to different features. For example, Wifi chipsets on
smartphones provide a variety of potentially-useful energy-performance knobs,
allowing the sleep interval to be altered, power levels adjusted, and
encodings to be changed. Unfortunately, however, not all chipsets support all
of these features, and it is also difficult to reason about how they combine
to affect meaningful performance characteristics such as latency or
throughput.

To address this challenge, Jouler incorporate a novel narrow waste for
implementing energy management utilizing idealized devices. An idealized
device simply trades off energy for some declared performance characteristic
in a smooth, though not necessarily linear, way. When actual components
register with Jouler, they indicate which energy-performance tradeoffs they
can perform. A DVFS memory chip, for example, would declare that it could
trade off throughput for energy usage. Policies in Jouler are then
implemented on top of these idealized components.

Translation between the idealized component and the actual component is done
by a component-specific piece of code that can be implemented by the driver
maintainer or anyone with specific knowledge about that piece of hardware.
\textbf{Thus, Jouler allows energy management policies to be written without
of hardware-specific details and transferred easily between devices.}
