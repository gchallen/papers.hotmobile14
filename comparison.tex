\section{Motes v. Discarded Phones}
\label{sec-comparison}

\renewcommand{\arraystretch}{1.2}
\begin{table*}[t]
\begin{threeparttable}
{\footnotesize
\begin{tabularx}{\textwidth}{rXXX}

&
\multicolumn{1}{c}{\textbf{Epic Mote}} &
\multicolumn{1}{c}{\textbf{Raspberry Pi Model B}} &
\multicolumn{1}{c}{\textbf{Discarded Nexus S 4G}} \\ \toprule

\textbf{Cost} &
\$69 &
\$35 &
\$22\tnote{1} \\ \midrule

\textbf{Microprocessor} &
4/8~MHz MSP~430 &
800~MHz\tnote{2} ARM1176JZ-F &
1~GHz\tnote{2} ARM Cortex A8 \\

\textbf{Memory} &
10~KB &
512~MB &
512~MB \\

\textbf{Storage} &
2~MB &
SD card sold separately\tnote{3} &
16~GB \\ \midrule

\textbf{Wireless Connectivity} &
802.15.4 &
None &
SMS, 3G data, Wifi (802.11 b/g/n), 4G WiMax \\ \midrule

\textbf{Packaging} &
Open circuit board &
Open circuit board &
Plastic case \\

\textbf{Human Interface} &
None &
LEDs, HDMI output &
480~x~800 pixel touch screen \\ \midrule

\textbf{Onboard Sensors} &
None &
None &
Location (GPS), accelerometer, gyroscope, proximity, compass, GPS, camera,
light \\

\textbf{Sensor Interface} &
8 ADC channels, 8 GPIO ports, OneWire &
2 USB ports, GPIO &
1 microUSB port \\ \midrule

\textbf{Operating System} &
TinyOS &
Linux &
Linux \\

\textbf{Programming} &
NesC &
Python, C &
Java, Android \\ \midrule


\textbf{Sleep Power} &
27~$\mu$W &
500~mW\tnote{4} &
4.2~mW \\

\textbf{Battery} &
None &
None &
Li-Ion 1500~mAh \\ \midrule

\end{tabularx}
}
{\footnotesize
\begin{tablenotes}
\item [1] Customer buyback price quoted by Sprint for a smartphone in good
condition.
\item [2] Processor is capable of dynamic voltage and frequency scaling
(DVFS).
\item [3] The cheapest 16~GB SD card we could locate on NewEgg cost \$9,
increasing the total cost of the Raspberry Pi to \$44.

\item [4] With the onboard hub removed~\cite{rpipower-url}.

\end{tablenotes}
}

\vspace*{-0.1in}

\caption{Comparison between potential sensing platforms. \textnormal{The
discarded Nexus~S~4G smartphone has multiple advantages compared with both
the Epic mote and the Raspberry Pi Model B.}}

\vspace*{0.1in}
\hrule
\vspace*{-0.2in}

\end{threeparttable}
\label{table-comparison}
\end{table*}


As part of determining whether discarded smartphones can replace sensor motes
for some sensor networking applications, we perform an attribute-by-attribute
comparison of the two options. We compare the Epic mote, a common sensor node
platform, with one discarded phone, the Samsung Nexus~S~4G  model that was
used for our preliminary feasibility study. Table~\ref{table-comparison}
presents numbers used in the discussion below, drawn from both datasheets and
experiments. As an additional comparison point, we also include numbers for
the Raspberry Pi Model B, since this is a popular, cheap and fairly-powerful
single-board computer.

\subsection{Cost and Availability}

We would expect the rapid turnover and high production volumes for consumer
devices like smartphones to cause their prices to start low and fall quickly,
and our data shows that this is the case. The Samsung Nexus~S was released in
2010 at \$529 unlocked, but only three years later Sprint offers customers
\$22 as a trade-in value for a returned device in good condition, making it
cheaper than both the Raspberry~Pi Model~B (\$35) and the Epic~Mote (\$69),
despite offering many more features as detailed below.

While both the Raspberry~Pi and Epic can be purchased new in unlimited
quantities, one concern about the use of discarded devices is availability.
However, as stated earlier the EPA estimates that 141~million mobile devices
were discarded in 2009. In several years, even if only half are Android
devices and only 10\% of those are in working condition, that still leaves
millions available for reuse. \textbf{Advantage: discarded smartphone.}

\subsection{Packaging and Human Interface}

Both the Epic~Mote and the Raspberry~Pi are shipped as bare circuit boards
and lack any human interface. In contrast, the discarded Nexus~S comes
packaged in plastic and features a familiar touch-screen interface.
\textbf{Advantage: discarded smartphone.}

\subsection{Sensors and Sensor Interface}

The Epic~Mote is designed as a sensor platform, and so includes multiple
sensor interfaces while integrating no on-board sensors. The Raspberry~Pi is
not designed for sensing applications but still exposes GPIO ports and a USB
interface allowing external sensors to be attached.

On the Nexus~S and other discarded phones the situation is more complicated.
On the plus side, many phones include multiple integrated sensors, although
the sensor suite is designed around smartphone and mobile computing
applications and its composition varies across devices. In addition, the
increasingly-ubquitous $\mu$USB port creates the possibility of adding
additional sensors, or the even-more-ubiquitous audio jack can be
``hijacked''~\cite{FIXME-audiohijack} for this purpose.

Unfortunately, our experience is that many smartphone USB controllers are not
designed for this purpose. If they can enter USB host mode at all, their
power consumption when acting as a host is prohibitive, possibly because they
are expecting to receive power when the USB cable is connected.\XXXnote{GWA:
Need more details from Aslak here.} We are investigating this problem in more
detail and hope to be able to develop software solutions on devices where the
hardware has the necessary capabilities.

Overally, while smartphones integrate more sensors, difficulties with
operating in USB host mode prevent them from fully taking advantage of their
interfaces when attaching external sensors. \textbf{Advantage: even.}

\subsection{Programming Environment}

Both the TinyOS~\cite{tinyos} and Contiki~\cite{contiki} sensor node
programming frameworks are notoriously difficult to learn and use, a
challenge that the sensor network research community has been trying to
address for years. In contrast, there are over 1~million applications hosted
on the Google Play store\XXXnote{GWA: Would be great to have a developer
count, rather than an app count, here, but I couldn't figure out how to get
the Play Store to cough that up.}  evidence of a large and growing Android
developer community built on pre-existing familiarity with the Java
programming language. We anticipate that the capabilities of these developers
can be harnessed in developing applications that reuse discarded phones as
sensors and in other ways. \textbf{Advantage: discarded smartphone.}

\subsection{Capabilities}

When comparing the core device capabilities, the discarded phone has the
clear advantage, with a three orders-of-magnitude faster processor, four
orders-of-magnitude more memory, and an one order-of-magnitude more storage.
The discarded phone actually has more memory than the Epic has Flash storage,
allowing it to cache information in RAM and avoid the energy overhead of
writing to or reading from Flash. While Flash has the benefit of persisting
across failures, motes tiny memory sizes typically cause data that could be
storage in RAM to be move to Flash simply due to lack of space. The Raspberry
Pi's core specifications are similar to the discarded smartphone, with the
exception that storage is not included and must be added separately at
additional cost.

\textbf{Advantage: discarded smartphone.}

\subsection{Connectivity}

\textbf{Advantage: discarded smartphone.}

\subsection{Power Consumption}

\textbf{Advantage: mote.}

\subsection{Ideal Uses}
